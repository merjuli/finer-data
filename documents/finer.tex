
%%%%%%%%%%%%%%%%%%%%%%% file template.tex %%%%%%%%%%%%%%%%%%%%%%%%%
%
% This is a general template file for the LaTeX package SVJour3
% for Springer journals.          Springer Heidelberg 2010/09/16
%
% Copy it to a new file with a new name and use it as the basis
% for your article. Delete % signs as needed.
%
% This template includes a few options for different layouts and
% content for various journals. Please consult a previous issue of
% your journal as needed.
%
%%%%%%%%%%%%%%%%%%%%%%%%%%%%%%%%%%%%%%%%%%%%%%%%%%%%%%%%%%%%%%%%%%%

%
%\RequirePackage{fix-cm}
%
%\documentclass{svjour3}                     % onecolumn (standard format)
%\documentclass[smallcondensed]{svjour3}     % onecolumn (ditto)
%\documentclass[smallextended]{svjour3}       % onecolumn (second format)
%\documentclass[twocolumn]{svjour3}          % twocolumn
%
%\smartqed  % flush right qed marks, e.g. at end of proof
%
%\usepackage{graphicx}
%
% \usepackage{mathptmx}      % use Times fonts if available on your TeX system
%
% insert here the call for the packages your document requires
%\usepackage{latexsym}
% etc.
%
% please place your own definitions here and don't use \def but
% \newcommand{}{}
%
% Insert the name of "your journal" with
% \journalname{myjournal}
%

%% ===============================================


%\RequirePackage{fix-cm}
%
%\documentclass{svjour3}                     % onecolumn (standard format)
%\documentclass[smallcondensed]{svjour3}     % onecolumn (ditto)
%\documentclass[smallextended]{svjour3}       % onecolumn (second format)
%\documentclass[twocolumn]{svjour3}          % twocolumn
%
%\smartqed  

\documentclass[11pt]{article}
\usepackage{times}
\usepackage{latexsym}
\usepackage{amsmath}
\usepackage{subfig, multicol}
\usepackage[all]{xy}
\usepackage[round]{natbib}
\usepackage{url}
%\usepackage[pdftex]{graphicx}
\usepackage[T1]{fontenc}
\usepackage{amsfonts}
%\usepackage{amsthm}
\usepackage{dsfont}
\usepackage{arydshln}


%% MINE

% \setlength{\parindent}{0pt}

\newcommand{\fixme}[1]{\textsl{[#1]}}

\DeclareMathOperator*{\argmax}{arg\,max}
\DeclareMathOperator*{\argmin}{arg\,min}
\DeclareMathOperator*{\argtop}{arg\,top}

%\newtheorem{definition}{Definition}
%\newtheorem{theorem}{Theorem}
%\newtheorem{lemma}{Lemma}

\begin{document}

\title{Statistical Named Entity Recognition Resources for Finnish}
\author{Teemu Ruokolainen \and Miikka Silfverberg \and Krister Lind\'en}



\maketitle


\begin{abstract}
\noindent 

\end{abstract}

\section{Introduction}
\label{sec: introduction}

Named entity recognition (NER) is a textual information extraction task, in which the aim is to locate and classify named entity expressions into pre-defined classes \citep{}. The set of classes typically include \textit{persons}, \textit{locations}, and \textit{organizations}, but can be extended with such classes as \textit{products}, \textit{events}, and \textit{temporal expressions}. The extracted entities can be utilized in a straightforward manner, for example, when indexing documents in an article database \citep{}. On the other hand, the extraction is often employed as a preprocessing step in a more complex language processing pipeline. For example, consider .

In this paper, we describe our work on building the essential resources for performing statistical named entity recognition for Finnish. Our main contributions are as follows. First, we present a corpus of XXX word tokens with a manually prepared named entity annotation. The corpus consists of technology related new articles and is freely available. Second, we present a statistical named entity recognition system for Finnish. The system learns to perform the recognition from the annotated corpus utilizing the efficient structured perceptron algorithm \citep{collins2002}. The system is bundled with the recently published Finnish morphological annotation toolkit, FinnPos \citep{silfverberg2016}, and is freely available.

The rest of the paper is organized as follows. We describe the corpus with manually prepared named entity annotation and the named entity recognition system in Sections \ref{sec: corpus} and \ref{sec: named entity recognizer}, respectively.
% In Section \ref{sec: corpus}, we describe the corpus with manually prepared named entity annotation. In Section \ref{sec: named entity recognizer}, we present the named entity recognizer. 
We then present experimental evaluation of the system in Section \ref{sec: experiments}. Finally, conclusion on the work are presented in Section \ref{sec: conclusions}.




\section{Corpus}
\label{sec: corpus}

In this section, we describe the text material, the set of named entity classes, and the annotation process.

\subsection{Text}
\label{sec: text}

The text material is extracted from the archives of Digitoday, a Finnish online technology news source.\footnote{See \url{www.digitoday.fi}.} The material covers a variety of technology related topics, such as business, science, and information security.  The material is extracted from articles published between years 2014 and 2015. The extracted text contains XXX word tokens in total. In addition to raw word forms, the corpus includes meta data describing the publishing date of the articles and if the word tokens belong to a headline, an ingress, or an article body. 


\subsection{Named Entities}
\label{sec: named entities}

In this section, we describe the named entity classes.



\paragraph{Person (PER}

Markable person names include:

\begin{itemize}

\item[1.] First names: e.g. Sauli, Barack
\item[2.] Family names: e.g. Niinist\"o, Obama
\item[3.] Aliases: e.g. DoctorClu

\end{itemize}



\paragraph{Location (LOC)}

Markable locations include:

\begin{itemize}

\item[1.] Continents: e.g. Eurooppa (Europe)
\item[2.] Countries: e.g. Suomi (Finland)
\item[3.] Cities: e.g. Helsinki
\item[4.] Planets: e.g. Mars
\item[5.] Buildings: e.g. Valkoinen talo (the White House)

\end{itemize}


\paragraph{Organization (ORG)}

Markable organizations include:

\begin{itemize}

\item[1.] Commercial companies: e.g. Nokia, Apple
\item[1.] Communities of people: e.g. Google Orkut
\item[1.] Education and research institutes: e.g. Turun Yliopisto
\item[1.] Political parties: e.g. Kokoomus (the National Coalition Party)
\item[1.] Public administration: e.g. Suomen hallitus (the Finnish Government), Euroopan Unioni (the European Union) % Ulkoministeri\"o (the Ministry for Foreign Affairs), Suomen hallitus (the Finnish Government)
\item[1.] News agencies/News services/Newspapers/Newsrooms/News sites/News blogs: e.g. Reuters, Helsingin Sanomat
\item[1.] Television network/station/channel: e.g. MTV3, FOX
\item[1.] Stock exchange: e.g. New Yorkin p\"orssi (New York Stock Exchange)
%\item[1.] Websites (not single web-links, see below): e.g. Amazon

\end{itemize}


\paragraph{Product (PROD)}

Markable products include:

\begin{itemize}

\item[1.] Laws: e.g. Patriot Act-laki (the Patriot Act law)
\item[1.] Networks (other than television network): e.g. Tor
\item[1.] Platforms: e.g. Google Play, Kickstarter
\item[1.] Projects/Programs: e.g. Vitja
\item[1.] Protocols: e.g. pop3, imap
\item[1.] Services/Platforms: e.g. Apple Store, Google Play
\item[1.] Software: e.g. Windows 10, Trojan-Banker.Win32.Chthonic
\item[1.] Vehicles/Vessels: e.g. Tesla Model S, Opportunity
\item[1.] Systems: e.g. Alipay-j\"arjestelm\"a
\item[1.] Technologies: e.g. 4G, HoloLens-teknologia
\item[1.] Works/Art: e.g. Tuntematon Sotilas, Hurt Locker
 
\end{itemize}


\paragraph{Event (EVENT)}

Markable events include:

\begin{itemize}

\item[1.] Expos: e.g. CES-messut
\item[1.] Explicitly marked events: e.g. Mobile World-tapahtuma

\end{itemize}


\paragraph{Temporal (TEMP)} Markable temporal expressions include:

\begin{itemize}

\item[1.] Nouns: e.g. torstai (Thursday), tulevaisuus (the future)
\item[2.] Noun Phrases: e.g. viimeiset kaksi vuotta (the last two years),  Runebergin p\"aiv\"a (Runeberg Day)
\item[3.] Adjectives: e.g. t\"am\"anhetkinen (current), kymmenvuotias (ten-year-old) 
\item[4.] Adjective Phrases: e.g. kymmenen tuntia pitk\"a (ten hours long), 
\item[5.] Adverbs: e.g. \"askett\"ain (recently), kuukausittain (monthly)
\item[6.] Adverb Phrases: e.g. kaksi viikkoa sitten (two weeks ago),

\end{itemize}

According to our definition, temporal expressions may contain adpositions. For example, the following are considered valid temporal expressions: ennen maanantaina (before Monday), tiistain j\"alkeen (after Tuesday). This practice is consistent with the Universal Dependencies guideline \cite{}. 

Following the above presentation, a wide range of temporal expressions are included in the annotation scheme. Note, however, that we omit such temporal expression-like subordinate clauses which do not explicitly denote a certain point in time, a duration, or a frequency. Consider, for example: Flappy  Bird -pelist\"a on tullut suuri mediat\"ahti sen j\"alkeen, kun kehitt\"aj\"a veti pelin pois Applen App Storesta (Flappy Bird has become a big media star after its developer withdrew the game from the Apple App Store).


\paragraph{Title (TITLE)}

Titles for people are markable if they appear in the immediate pre-context of proper names. For example, in the following, markable titles are bolded: \textbf{presidentti} Sauli Niinist\"o (\textbf{president} Sauli Niinist\"o), \textbf{perustaja} ja \textbf{varapuheenjohtaja} Ville Oksanen (\textbf{founder} and \textbf{vice president} Ville Oksanen).   





%\paragraph{URL} Markable 

%In general, temporal expressions refer to expressions denoting, for example, a certain point in time, a duration, or a frequency. In the following, we address some more detailed aspects concerning their annotation. 

%First, to be markable, the syntactic head of the expression must be an appropriate lexical trigger. A lexical trigger is a word or numeric expression, the meaning of which conveys a temporal unit or concept, such as ''day'', ''month'', or ''year''. Additionally, one must be able to orient the trigger on a timeline, or at least orient it with relation to a time (past, present, future). As for related work, this general guideline is identical with the TIMEX3 annotation standard \citep{verhagen2010}.

%Second, temporal expressions may contain adpositions, that is, we consider the adpositions to be subordinate to the head words. For example, the following are considered valid temporal expressions: ennen maanantaina (before Monday), tiistain j\"alkeen (after Tuesday). As for related work, this practice is consistent with the Universal Dependencies guideline \cite{}, while in TIMEX3 standard the adpositions are not included in the temporal expression extents.



%Furthermore, in Finnish, adpositions are often included in the expressions via inflection. For example, consider the English expression "during the summer" which in Finnish can be expressed in two ways as in "kes\"all\"a" or by using postposition as in "kes\"an aikana". T 

%In order to mark temporal expressions, we adopt the TIMEX3 annotation guideline specified for the SemEval-2010: Tempeval-2 evaluation task for English \citep{verhagen2010}.\footnote{The TIMEX3 guideline is available at \url{http://www.timeml.org/tempeval2/tempeval2-trial/guidelines/timex3guidelines-072009.pdf}} The guideline contains the following types of temporal expressions (TIMEXes): 
 
%To be markable, the syntactic head of the expression must be an appropriate lexical trigger. Each lexical trigger is a word or numeric expression whose meaning conveys a temporal unit or concept, such as ''day'' or ''monthly''. Furthermore, to be a trigger, the referent must be able to be oriented on a timeline, or at least oriented with relation to a time (past, present, future). These basic constraints are adopted here as presented in the TIMEX2 guideline, an earlier version of the TIMEX3 standard.\footnote{The TIMEX2 guideline is available at \url{https://www.ldc.upenn.edu/sites/www.ldc.upenn.edu/files/english-timex2-guidelines-v0.1.pdf}} 

%Note that \citet{verhagen2010} restricts the extents (or spans) of the expressions so that they can not begin with a Preposition or a clause of any type. For example, in the following, only the bolded parts are considered valid English temporal expressions: before \textbf{Thursday}, in \textbf{the morning}, after the strike ended on \textbf{Thursday}, over \textbf{the last 2 years}. As for Finnish, we extend this rule to consider Postpositions as well as Prepositions. Consequently, in the following, only the bolded parts are markable Finnish temporal expressions: ennen \textbf{torstaita} (before Thursday), \textbf{torstain} j\"alkeen (after Thursday). However, we allow the Prepositions and Postpositions if they are included in the expression via inflection. For example, we consider the following valid Finnish temporal expressions: \textbf{torstaina} (on Thursday), \textbf{kes\"all\"a} (in the summer).

%Second, according to \citep{verhagen2010}, 





\subsection{Annotation Process}

In order to annotate the text described in Section \ref{sec: text} using the named entities described in Section \ref{sec: named entities}, we first divide the text into two parts, training and test sections. In the training section, we include all articles published during 2014, while the test section consists of articles published in 2015. The resulting training and test sections contain xx and yy word tokens, respectively. Subsequently, the training section is annotated by a single annotator (Annotator A). Meanwhile, annotation of the test section is carried out by three annotators (Annotator A, Annotator B, and Annotator C). Specifically, the section is first annotated independently by Annotators A and B. Subsequently, found conflicts are resolved using the majority vote principle by Annotator C. The purpose of this three-stage procedure is to ensure a high test set quality.

% Specifically, the section is first annotated by Annotator A. Subsequently, the annotation is proof read by Annotator B. Found conflicts are then resolved using the majority vote principle by all Annotators A, B, and C.


\subsection{Statistics}

The resulting training and test sections contain xx and yy articles, or xx and yy word tokens, respectively. The counts of each named entity class in these sections are presented in Table \ref{tab: statistics}.


\begin{table}[h!]
%\begin{small}
\begin{center}
\begin{tabular}{lcc} 
\hline
\noalign{\smallskip}
Training & count & \% \\ 
\hline
\noalign{\smallskip}
PER & & \\
LOC & &  \\
ORG & & \\
PROD & & \\
EVENT & & \\
TITLE & & \\
TEMP & & \\
\hline
\noalign{\smallskip}
TOTAL & & \\
& & \\
\hline
\noalign{\smallskip}
Test & count & \% \\ 
\hline
\noalign{\smallskip}
PER & & \\
LOC & &  \\
ORG & & \\
PROD & & \\
EVENT & & \\
TITLE & & \\
TEMP & & \\
\hline
\noalign{\smallskip}
TOTAL & & \\\end{tabular}
\end{center}
% \caption{Statistics of the training and test sets including total number of tokens and named entity classes.}
\caption{Counts and relative portions of named entity classes in training and test corpora.}
\label{tab: statistics} 
\end{table}



% Specifically, the section is first annotated independently by Annotators A and B. Subsequently Subsequently, the annotations are compared by Annotator C. In case of conflicts

%\subsection{Comments and Known Issues}

%\paragraph{Organization versus Product} 1) For example, consider the two related but distinct meanings of Youtube in "Katsoa videoita Youtubesta" ja "Youtube on Googlen omistama yhti\"o". These are disambiguated according to context. 2) Websites are most often used in organisation-sense (Amazon.com, Verkkokauppa.com) but can be also used in product-sense ("Amazon.com-sivusto kaatui keskiy\"oll\"a."). The product-sense is chosen if appropriate.

%\paragraph{Organization versus Location} For example, consider India as a geographical location in "" and India as a governmental organization in "".

%\paragraph{Explicit naming} For example, consider "Nokia N1-niminen puhelin".



% \section{A Part-of-Speech Tagger for Finnish}
\section{Named Entity Recognizer}
\label{sec: named entity recognizer}

In this section, we present our statistical named entity recognizer. The recognizer is trained using the annotated corpus described in Section \ref{sec: corpus} and can subsequently be applied to extract named entities in a running text. 

\subsection{Overview}

The system utilizes two main components.

\begin{itemize}

\item[1.] We interpret the NER task as a sequential tagging task and perform the tagging using the averaged perceptron algorithm \citep{collins2002}.

\item[2.] In order to alleviate data sparseness caused by the rich morphology of Finnish, we utilize the FinnPos system \citep{silfverberg2016}, a recently published morphological and lemmatization toolkit for Finnish. FinnPos contains two separate tagging and lemmatization models learned from Finnish Turku Dependency Treebank (TDT) \citep{haverinen2014} and FinnTreeBank (FTB) \citep{voutilainen2011}. Thus, any given sentence can be readily tagged and lemmatized by FinnPos using two varying annotation schemes.

%\subsection{FinnPos}
%The FinnPos system \citep{silfverberg2015} is an open-source morphological tagging and lemmatization toolkit for Finnish. The morphological tagging component of the system is based on the averaged structured perceptron classifier \citep{collins2002}. The perceptron learning and decoding is accelerated using a combination of two approximations, namely, beam search and a model cascade. Meanwhile, the lemmatization is performed using a combination of OMorFi, an open-source morphological analyzer for Finnish \citep{pirinen2008}, and a statistical lemmatization model. FinnPos contains two separate tagging and lemmatization models learned from Finnish Turku Dependency Treebank (TDT) \citep{} and FinnTreeBank (FTB) \citep{}. In other words, any given sentence can be readily tagged and lemmatized by FinnPos using two varying annotation schemes. 

 \end{itemize}

%Given a sentence $x = (x_1, \dots, x_{|x|})$ and a label sequence $y = (y_1, \dots, y_{|x|})$, the structured perceptron classifier assigns the pair $(x,y)$ a score
%
%\begin{eqnarray}
%\text{score}(x,y; \boldsymbol{w}) & = & \sum_{i=n}^{|x|} \boldsymbol{w} \cdot \boldsymbol{\phi}(y_{i-n}, \dots, y_i, x, i) \, ,						
%\label{eq: perceptron}
%\end{eqnarray}
%
%where $n$ denotes the model order, $\boldsymbol{w}$ the model parameter vector, and $\boldsymbol{\phi}$ the feature extraction function. The word forms $x_i$ are assigned labels from a potentially large label set $\mathcal{Y}$, that is, $y_i \in \mathcal{Y}$ for all $i = 1, \dots, |x|$. As shown in Table \ref{tab: tdt ftb}, for TDT and FTB, the label sets $\mathcal{Y}$ contain roughly 2,000 and 1,400 morphological tags, respectively.

\subsection{Feature Extraction}

The appeal of the perceptron classifier \citep{collins2002} lies in its capability of utilizing rich, overlapping feature sets when mapping a sentence to a corresponding tag sequence. Specifically, each word/label position $i$ in a given sentence is associated with  a set of features describing the input sentence. Our basic feature set contains the following features:

\begin{itemize}
\item[1.] Bias (always active irrespective of input).
\item[2.] Word forms $x_{i-2}, \dots, x_{i+2}$.
\item[3.] Prefixes and suffixes of the word form $x_i$ up to length $\delta_{affix}$.
\item[4.] The brief and long generalized forms of the word forms $x_{i-2}, \dots, x_{i+2}$ following \citet{collins2002}.
\end{itemize}

Moreover, we augment the word tokens of each processed sentence with morphological tags and lemmas by applying both TDT and FTB models of the FinnPos system. As a result, each training word token is associated with two morphological tags and two lemmas. Subsequently, we expand our feature set using the following features (for both TDT and FTB separately):

\begin{itemize}
\item[5.] The lemmas for word forms $x_{i-2}, \dots, x_{i+2}$. 
\item[6.] The sub-lemmas for word forms $x_{i-2}, \dots, x_{i+2}$. 
\item[7.] The morphological tags for word forms $x_{i-2}, \dots, x_{i+2}$.
\item[8.] The morphological sub-tags for word forms $x_{i-2}, \dots, x_{i+2}$.
\end{itemize}

\fixme{Describe sub-lemmas and sub-tags} 

\subsection{Implementation Details}
\label{sec: implementation details}


\section{Experiments}
\label{sec: experiments}

In this section, we present the evaluation of the named entity recognition system described in Section \ref{sec: named entity recognizer} on the corpus described in Section \ref{sec: corpus}.

\subsection{Data}
\label{sec: data}

We employ the training and test set division of the data described in Section \ref{sec: corpus}. A separate development set is formed from the training data by extracting every 10th article, starting from the 10th article. (The articles are separated using the headline information referred to in Section \ref{sec: corpus}.) The resulting training, development, and test set sizes are presented in Table \ref{tab: set sizes}.

\begin{table}[t!]
%\begin{small}
\begin{center}
\begin{tabular}{lcc} 
 & \#articles & \#tokens \\
\hline
\noalign{\smallskip}
training & x & y  \\
development  & x & y  \\
test & x & y \\
\end{tabular}
\end{center}
%\end{small}
\caption{Training, development, and test set sizes.}
\label{tab: set sizes}
\end{table}


\subsection{Reference Systems OR The FiNER System}
\label{sec: reference systems}

As a reference system, we use the rule-based named entity recognized developed at Helsinki University.  

OR 

In addition to the statistical named entity recognizer described in Section \ref{sec: named entity recognizer}, we evalute the rule-based named entity recognizer, FiNER, developed at the Helsinki University.


\subsection{Evaluation and Error Analysis}
\label{sec: evaluation}

We evaluate the system using the standard \textbf{precision} (the number of correctly recognized entities divided by the number of all recognized entities), \textbf{recall} (the number of correctly recognized entities divided by the the number of all annotated entities in data), and \textbf{F-score} (the geometric mean of precision and recall).

In order to gain more insight on the type of errors produced by the system, we analyze a \textbf{confusion matrix} of entities. In the matrix, we consider all entities which are located correctly, i.e., have a correct span over word tokens, by the system. The elements of the matrix then indicate which entity-spans were correctly detected but then assigned an incorrect entity-label. 



\subsection{Results}
\label{sec: results}


\paragraph{Precision, Recall, and F-scores}
% \label{sec: results1}

\begin{table}[t!]
%\begin{small}
\begin{center}
\begin{tabular}{cccc} 
%\hline
%\noalign{\smallskip}
%\multicolumn{1}{c}{}  & \multicolumn{2}{c}{tag acc.} &  \multicolumn{2}{c}{lemma acc.} & \multicolumn{2}{c}{train. time} &  \multicolumn{1}{c}{dec. speed (tok/s)} \\
Entity class & pre & rec & F \\
\hline
\noalign{\smallskip}
LOC  & x & y & z  \\
ORG  & x & y & z  \\
PER  & x & y & z  \\
PROD  & x & y & z  \\
TITLE  & x & y & z  \\
EVENT  & x & y & z  \\
TEMP  & x & y & z  \\
\hline
\noalign{\smallskip}
Any & x & y & z  \\
\end{tabular}
\end{center}
%\end{small}
\caption{Precision, recall, and F-scores for each named entity class.}
\label{tab: precision recall and f-scores}
\end{table}


\paragraph{Confusion Matrix}
% \label{sec: results2}


\begin{table}[t!]
%\begin{small}
\begin{center}
\begin{tabular}{cccccccc} 
%\hline
%\noalign{\smallskip}
%\multicolumn{1}{c}{}  & \multicolumn{2}{c}{tag acc.} &  \multicolumn{2}{c}{lemma acc.} & \multicolumn{2}{c}{train. time} &  \multicolumn{1}{c}{dec. speed (tok/s)} \\
& LOC & ORG & PERSON & PROD & TITLE & EVENT & TEMP \\
\hline
\noalign{\smallskip}
LOC  & - & & & & & &  \\
ORG  & & - & & & & &  \\
PER  & & & - & & & &  \\
PRO  & & & & - & & &  \\
TITLE  & & & & & - & &  \\
EVENT  & & & & & & - &  \\
TEMP  & & & & & & &  - \\
\end{tabular}
\end{center}
%\end{small}
\caption{Confusion matrix for named entity classes. Each element represent the number of times a row-entity was mislabeled for a column-label. For example, a person-entity was mistakenly labeled as a person-entity X times.}
\label{tab: confusion matrix}
\end{table}

\paragraph{Time Requirements}

The training time of the system was XX minutes. Subsequent to training, the system's decoding speed was XXXXXX tokens per second, resulting in a decoding time of XX minutes in total. The experiments were run on a standard desktop computer XX XX XX.


\subsection{Discussion}
\label{sec: discussion}

In the following, we 



\section{Conclusions}
\label{sec: conclusions}





%%%%%%%%%%% The bibliography starts:
\newpage
\bibliographystyle{plainnat}
%\bibliographystyle{spbasic}
\bibliography{finer.bib}









\end{document}














