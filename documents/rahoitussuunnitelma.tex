

 \documentclass[12pt,a4paper,finnish,oneside]{article}
 
%\usepackage[latin1]{inputenc}
\usepackage[utf8]{inputenc} 
 \usepackage[T1]{fontenc}
 \usepackage[finnish]{babel}
 \usepackage{url}


\newcommand{\fixme}[1]{\textsl{[#1]}}




\begin{document}


\title{Automaattisen nimentunnistimen kehittäminen suomen kielelle: Rahoitussuunnitelma}
\author{Teemu Ruokolainen}


\maketitle

Tutkimushanke toteutetaan 1.2.2016 ja 28.2.2017 välisenä aikana Helsingin Yliopiston Nykykielten laitoksella. Hankkeen kokonaiskustannukseksi tälle ajalle arvioidaan 15.600 euroa (1.200 euroa / kk).

% Hankkeen kokonaiskustannukseksi tälle ajalle arvioidaan 41028 euroa. Summa koostuu hakijan palkkakustannuksista (3156 euroa/kk, YPJ/Opetus- ja tutkimushenkilöstö, vaatimustaso OV04, suoritustaso OS06). Nykykielten laitos on sitoutunut maksamaan tästä summasta 50\% (20514 euroa). Tässä hakemuksessa esitetty 20514 euron apuraha muodostaa siis hyväksyttynä 50\% kokonaiskustannuksista.


% Hankkeen kokonaiskustannukseksi vuonna 2016 arvioidaan 36000 euroa. Summa koostuu hakijan palkkakustannuksista (3156,29 euroa/kk). FinClarin-projekti () on sitoutunut maksamaan summasta 50\% (18000 euroa). Tässä hakemuksessa esitetty 18000 euroa muodostaa siis loput 50\% kokonaiskustannuksista.



%Nimentunnistaja kehitetään hyödyntäen modernia tilastollista koneoppimismetodologiaa. Kehitysprosessi jakaantuu kahteen osaan. Ensimmäisessä osassa ihmisasiantuntija paikantaa ja merkitsee nimet tekstiaineistoon. Lähtökohtaisena tekstiaineistona käytetään uutissivusto DigiTodaysta vuosina 2014 ja 2015 haettuja artikkeleita. Toisessa osassa merkittyä aineistoa käytetään tilastollisen tunnistajan opettamiseen. Ihmisen muodostaman opetusaineiston lisäksi tunnistaja hyödyntää Helsingin Yliopistossa aikasemmin kehitettyä tilastollista morfologista suomen kielen jäsennintä, FinnPosia \citep{}, sekä sääntöpohjaista suomen kielen nimentunnistajaa, FiNer:iä \citep{}. Sääntöpohjaisen FiNer-tunnistajan vahvuus piilee sen hyvässä tarkkuudessa (precision) mutta heikkous huonossa kattavuudessa (recall). Menetelmän hyvä tarkkuus merkitsee sitä, että kaikki tunnistetut termit ovat suurella todennäköisyydellä oikeita nimiä. Huono kattavuus taas merkitsee sitä, että suuri osa nimistä jää tunnistamatta suurella todennäköisyydellä. Tilastolliset menetelmät sitä vastoin saavuttavat sekä hyvän tarkkuuden että hyvän kattavuuden. Tämän vuoksi parhaimmat nimentunnistimet, kuten alunperin englannille kehitetty Stanford Named Entity Recognizer \citep{}, hyödyntävät tilastollista koneoppimismetodologiaa. Tilastollista lähestymistapaa hyödyntävää tunnistajaa on myös helpompi laajentaa muihin tekstilajeihin verrattuna sääntöpohjaiseen menetelmään.


%Automaattisen kielenkäsittelyn tutkimuksessa nimentunnistus on ollut pitkään kehityksen kohde. Esimerkiksi englannille ensimmäiset modernia tilastollista metodologiaa käyttävät tunnistustyökalut (mm. Stanford Named Entity Recognizer) julkaistiin jo 2000-luvun puolivälissä. On kuitenkin tyypillistä, että pienemmillä kielialueilla kieliresurssien kehittäminen ja julkaiseminen tapahtuu jäljessä. Esimerkiksi ensimmäinen englannin kielisiä resursseja vastaava ruotsinkielinen nimentunnistustyökalu (SweNER) julkaistiin vuonna 2013. Vastaavaa resurssia ei ole julkaistu suomelle.


\end{document}


