
%%%%%%%%%%%%%%%%%%%%%%% file template.tex %%%%%%%%%%%%%%%%%%%%%%%%%
%
% This is a general template file for the LaTeX package SVJour3
% for Springer journals.          Springer Heidelberg 2010/09/16
%
% Copy it to a new file with a new name and use it as the basis
% for your article. Delete % signs as needed.
%
% This template includes a few options for different layouts and
% content for various journals. Please consult a previous issue of
% your journal as needed.
%
%%%%%%%%%%%%%%%%%%%%%%%%%%%%%%%%%%%%%%%%%%%%%%%%%%%%%%%%%%%%%%%%%%%

%
%\RequirePackage{fix-cm}
%
%\documentclass{svjour3}                     % onecolumn (standard format)
%\documentclass[smallcondensed]{svjour3}     % onecolumn (ditto)
%\documentclass[smallextended]{svjour3}       % onecolumn (second format)
%\documentclass[twocolumn]{svjour3}          % twocolumn
%
%\smartqed  % flush right qed marks, e.g. at end of proof
%
%\usepackage{graphicx}
%
% \usepackage{mathptmx}      % use Times fonts if available on your TeX system
%
% insert here the call for the packages your document requires
%\usepackage{latexsym}
% etc.
%
% please place your own definitions here and don't use \def but
% \newcommand{}{}
%
% Insert the name of "your journal" with
% \journalname{myjournal}
%

%% ===============================================


%\RequirePackage{fix-cm}
%
%\documentclass{svjour3}                     % onecolumn (standard format)
%\documentclass[smallcondensed]{svjour3}     % onecolumn (ditto)
%\documentclass[smallextended]{svjour3}       % onecolumn (second format)
%\documentclass[twocolumn]{svjour3}          % twocolumn
%
%\smartqed  

\documentclass[11pt]{article}
\usepackage{times}
\usepackage{latexsym}
\usepackage{amsmath}
\usepackage{subfig, multicol}
\usepackage[all]{xy}
\usepackage[round]{natbib}
\usepackage{url}
%\usepackage[pdftex]{graphicx}
\usepackage[T1]{fontenc}
\usepackage{amsfonts}
%\usepackage{amsthm}
\usepackage{dsfont}
\usepackage{arydshln}


%% MINE

% \setlength{\parindent}{0pt}

\newcommand{\fixme}[1]{\textsl{[#1]}}

\DeclareMathOperator*{\argmax}{arg\,max}
\DeclareMathOperator*{\argmin}{arg\,min}
\DeclareMathOperator*{\argtop}{arg\,top}

%\newtheorem{definition}{Definition}
%\newtheorem{theorem}{Theorem}
%\newtheorem{lemma}{Lemma}

\begin{document}

\title{Statistical Named Entity Recognition Resources for Finnish}
\author{Teemu Ruokolainen \and Miikka Silfverberg \and Krister Lind\'en}



\maketitle


\begin{abstract}
\noindent 

\end{abstract}

\section{Introduction}
\label{sec: introduction}

Named entity recognition (NER) is a textual information extraction task, in which the aim is to locate and classify named entity expressions into pre-defined classes \citep{}. The set of classes typically include \textit{persons}, \textit{locations}, and \textit{organizations}, but can be extended with such classes as \textit{products}, \textit{events}, and \textit{temporal expressions}. The extracted entities can be useful, for example, when indexing documents in an article database \citep{}. On the other hand, the extraction is often employed as a preprocessing step in a more complex language processing pipeline. For example, consider .

In this paper, we describe our work on building the essential resources for performing statistical named entity recognition for Finnish. The main contributions include:

\begin{itemize}
\item[1.] We present a corpus of XXX word tokens with a manually prepared named entity annotation. The corpus is freely available.

\item[2.] We present a statistical named entity recognition toolkit for Finnish. The toolkit is built on the recently published Finnish morphological annotation toolkit, FinnPos, and is freely available. 

\end{itemize}

The rest of the paper is organized as follows.




\section{Data}
\label{sec: data}

\subsection{Text}
\label{sec: text}

The text material is extracted from the archives of Digitoday, a Finnish online technology news source. The material covers a variety of technology related topics, such as business, science, and information security.  The material is extracted from articles published between years 2014 and 2015. The extracted text contains 250,000 word tokens in total. In addition to raw word forms, the corpus includes meta data describing if the word tokens belong to a headline, an ingress, or article body. 


\subsection{Named Entities}
\label{sec: named entities}

In this section, we describe the named entity classes.



\paragraph{Person}

Markable person names include:

\begin{itemize}

\item[1.] First names: e.g. Sauli, Barack
\item[2.] Family names: e.g. Niinist\"o, Obama
\item[3.] Aliases: e.g. DoctorClu

\end{itemize}



\paragraph{Location}

Markable locations include:

\begin{itemize}

\item[1.] Continents: e.g. Eurooppa (Europe)
\item[2.] Countries: e.g. Suomi (Finland)
\item[3.] Cities: e.g. Helsinki
\item[4.] Planets: e.g. Mars
\item[5.] Buildings: e.g. Valkoinen talo (the White House)

\end{itemize}


\paragraph{Organization}

Markable organizations include:

\begin{itemize}

\item[1.] Commercial companies: e.g. Nokia, Apple
\item[1.] Communities of people: e.g. Google Orkut
\item[1.] Education and research institutes: e.g. Turun Yliopisto
\item[1.] Political parties: e.g. Kokoomus (the National Coalition Party)
\item[1.] Public administration: e.g. Ulkoministeri\"o (the Ministry for Foreign Affairs), Suomen hallitus (the Finnish Government)
\item[1.] News agencies/News services/Newspapers/Newsrooms/News sites/News blogs: e.g. Reuters, Helsingin Sanomat
\item[1.] Television network/station/channel: e.g. MTV3, FOX
\item[1.] Stock exchange: e.g. New Yorkin p\"orssi (New York Stock Exchange)
%\item[1.] Websites (not single web-links, see below): e.g. Amazon

\end{itemize}


\paragraph{Product}

Markable products include:

\begin{itemize}

\item[1.] Laws: e.g. Patriot Act-laki (the Patriot Act law)
\item[1.] Networks (other than television network): e.g. Tor
\item[1.] Platforms: e.g. Google Play, Kickstarter
\item[1.] Projects/Programs: e.g. Vitja
\item[1.] Protocols: e.g. pop3, imap
\item[1.] Services/Platforms: e.g. Apple Store, Google Play
\item[1.] Software: e.g. Windows 10, Trojan-Banker.Win32.Chthonic
\item[1.] Space probes: e.g. Opportunity
\item[1.] Systems: e.g. Alipay-j\"arjestelm\"a
\item[1.] Technogies: e.g. 4G, HoloLens-teknologia
\item[1.] Works/Art: e.g. Tuntematon Sotilas, Hurt Locker
 
\end{itemize}



\paragraph{Event}

Markable events include:

\begin{itemize}

\item[1.] Expos: e.g. CES-messut
\item[1.] Explicitly marked events: e.g. Mobile World-tapahtuma

\end{itemize}



\paragraph{Temporal} Markable temporal expressions include:

\begin{itemize}

\item[1.] Nouns: e.g. torstai (Thursday), tulevaisuus (the future)
\item[2.] Noun Phrases: e.g. viimeiset kaksi vuotta (the last two years),  Runebergin p\"aiv\"a (Runeberg Day)
\item[3.] Adjectives: e.g. t\"am\"anhetkinen (current), kymmenvuotias (ten-year-old) 
\item[4.] Adjective Phrases: e.g. kymmenen tuntia pitk\"a (ten hours long), 
\item[5.] Adverbs: e.g. \"askett\"ain (recently), kuukausittain (monthly)
\item[6.] Adverb Phrases: e.g. kaksi viikkoa sitten (two weeks ago),

\end{itemize}

According to our definition, temporal expressions may contain adpositions. For example, the following are considered valid temporal expressions: ennen maanantaina (before Monday), tiistain j\"alkeen (after Tuesday). This practice is consistent with the Universal Dependencies guideline \cite{}. According to the above criteria, a wide range of temporal expressions are included in the annotation scheme. Note, however, that we omit such temporal expression-like subordinate clauses which do not explicitly denote a certain point in time, a duration, or a frequency. Consider, for example: Flappy  Bird -pelist\"a on tullut suuri mediat\"ahti sen j\"alkeen, kun kehitt\"aj\"a veti pelin pois Applen App Storesta (Flappy Bird has become a big media star after its developer withdrew the game from the Apple App Store).


\paragraph{Title}

Titles for people are markable if they appear in the immediate pre-context of proper names. For example, in the following, markable titles are bolded: \textbf{presidentti} Sauli Niinist\"o (\textbf{president} Sauli Niinist\"o), \textbf{perustaja} ja \textbf{varapuheenjohtaja} Ville Oksanen (\textbf{founder} and \textbf{vice president} Ville Oksanen).   




%\paragraph{URL} Markable 

%In general, temporal expressions refer to expressions denoting, for example, a certain point in time, a duration, or a frequency. In the following, we address some more detailed aspects concerning their annotation. 

%First, to be markable, the syntactic head of the expression must be an appropriate lexical trigger. A lexical trigger is a word or numeric expression, the meaning of which conveys a temporal unit or concept, such as ''day'', ''month'', or ''year''. Additionally, one must be able to orient the trigger on a timeline, or at least orient it with relation to a time (past, present, future). As for related work, this general guideline is identical with the TIMEX3 annotation standard \citep{verhagen2010}.

%Second, temporal expressions may contain adpositions, that is, we consider the adpositions to be subordinate to the head words. For example, the following are considered valid temporal expressions: ennen maanantaina (before Monday), tiistain j\"alkeen (after Tuesday). As for related work, this practice is consistent with the Universal Dependencies guideline \cite{}, while in TIMEX3 standard the adpositions are not included in the temporal expression extents.



%Furthermore, in Finnish, adpositions are often included in the expressions via inflection. For example, consider the English expression "during the summer" which in Finnish can be expressed in two ways as in "kes\"all\"a" or by using postposition as in "kes\"an aikana". T 

%In order to mark temporal expressions, we adopt the TIMEX3 annotation guideline specified for the SemEval-2010: Tempeval-2 evaluation task for English \citep{verhagen2010}.\footnote{The TIMEX3 guideline is available at \url{http://www.timeml.org/tempeval2/tempeval2-trial/guidelines/timex3guidelines-072009.pdf}} The guideline contains the following types of temporal expressions (TIMEXes): 
 
%To be markable, the syntactic head of the expression must be an appropriate lexical trigger. Each lexical trigger is a word or numeric expression whose meaning conveys a temporal unit or concept, such as ''day'' or ''monthly''. Furthermore, to be a trigger, the referent must be able to be oriented on a timeline, or at least oriented with relation to a time (past, present, future). These basic constraints are adopted here as presented in the TIMEX2 guideline, an earlier version of the TIMEX3 standard.\footnote{The TIMEX2 guideline is available at \url{https://www.ldc.upenn.edu/sites/www.ldc.upenn.edu/files/english-timex2-guidelines-v0.1.pdf}} 

%Note that \citet{verhagen2010} restricts the extents (or spans) of the expressions so that they can not begin with a Preposition or a clause of any type. For example, in the following, only the bolded parts are considered valid English temporal expressions: before \textbf{Thursday}, in \textbf{the morning}, after the strike ended on \textbf{Thursday}, over \textbf{the last 2 years}. As for Finnish, we extend this rule to consider Postpositions as well as Prepositions. Consequently, in the following, only the bolded parts are markable Finnish temporal expressions: ennen \textbf{torstaita} (before Thursday), \textbf{torstain} j\"alkeen (after Thursday). However, we allow the Prepositions and Postpositions if they are included in the expression via inflection. For example, we consider the following valid Finnish temporal expressions: \textbf{torstaina} (on Thursday), \textbf{kes\"all\"a} (in the summer).

%Second, according to \citep{verhagen2010}, 





\subsection{Annotation Process}

In order to annotate the text described in Section \ref{sec: text} using the named entities described in Section \ref{sec: named entities}, we first divide the text in to two parts, training and test sections. In the training section, we include all articles published during 2014 while the test section consists of articles published in 2015. The resulting training and test sections contain xx and yy word tokens, respectively. Subsequently, the training section is annotated by a single annotator (Annotator A). Meanwhile, annotation of the test section is carried out by three annotators (Annotator A, Annotator B, and Annotator C). Specifically, the section is first annotated independently by Annotators A and B. Subsequently, found conflicts are resolved using the majority vote principle by Annotator C.

% Specifically, the section is first annotated by Annotator A. Subsequently, the annotation is proof read by Annotator B. Found conflicts are then resolved using the majority vote principle by all Annotators A, B, and C.


\subsection{Statistics}

The resulting training and test sections contain xx and yy word tokens, respectively. The number of each named entity class in both sections are presented in Table \ref{tab: statistics}.


\begin{table}[h!]
%\begin{small}
\begin{center}
\begin{tabular}{lcc} 
\hline
\noalign{\smallskip}
Training & count & \% \\ 
\hline
\noalign{\smallskip}
PERSON & \\
LOCATION & \\
ORGANIZATION & \\
PRODUCT & \\
EVENT & \\
TITLE & \\
TEMPORAL & \\
\hline
\noalign{\smallskip}
Test & count & \% \\ 
\hline
\noalign{\smallskip}
PERSON & \\
LOCATION & \\
ORGANIZATION & \\
PRODUCT & \\
EVENT & \\
TITLE & \\
TEMPORAL & \\
\end{tabular}
\end{center}
% \caption{Statistics of the training and test sets including total number of tokens and named entity classes.}
\caption{Counts and relative portions of named entity classes in training and test corpora.}
\label{tab: statistics} 
\end{table}



% Specifically, the section is first annotated independently by Annotators A and B. Subsequently Subsequently, the annotations are compared by Annotator C. In case of conflicts

%\subsection{Comments and Known Issues}

%\paragraph{Organization versus Product} 1) For example, consider the two related but distinct meanings of Youtube in "Katsoa videoita Youtubesta" ja "Youtube on Googlen omistama yhti\"o". These are disambiguated according to context. 2) Websites are most often used in organisation-sense (Amazon.com, Verkkokauppa.com) but can be also used in product-sense ("Amazon.com-sivusto kaatui keskiy\"oll\"a."). The product-sense is chosen if appropriate.

%\paragraph{Organization versus Location} For example, consider India as a geographical location in "" and India as a governmental organization in "".

%\paragraph{Explicit naming} For example, consider "Nokia N1-niminen puhelin".



% \section{A Part-of-Speech Tagger for Finnish}
\section{Named Entity Recognizer}
\label{sec: named entity recognizer}

In this section, we present how we extend the FinnPos system \citep{silfverberg2015} for named entity recognition.


\subsection{FinnPos}

The FinnPos system \citep{silfverberg2015} is an open-source morphological tagging and lemmatization toolkit for Finnish. The morphological tagging component of the system is based on the averaged structured perceptron classifier \citep{collins2002}. The perceptron learning and decoding is accelerated using a combination of two approximations, namely, beam search and a model cascade. Meanwhile, the lemmatization is performed using OMorFi, an open-source morphological analyzer for Finnish \citep{pirinen2008}. In order to lemmatize word forms unknown to OMorFi, FinnPos implements a statistical lemmatization model. FinnPos contains two separate models learned from Finnish Turku Dependency Treebank (TDT) \citep{} and FinnTreeBank (FTB) \citep{}. In other words, any given sentence can be readily tagged and lemmatized by FinnPos using two varying annotation schemes. 


\subsection{Extending FinnPos for Named Entity Recognition}

In order to extend the FinnPos system for named entity recognition, we adapt the following procedure. First, given training data, we augment the word tokens in the data with morphological tags and lemmas by applying both TDT and FTB models of the FinnPos system. As a result, each training word token is assigned two morphological tags and two lemmas. Second, we interpret the named entity recognition task as a sequential tagging problem with the classic BIO-annotation scheme and apply a second-order averaged perceptron classifier \citep{collins2003}. Specifically, the perceptron classifier employs the following feature extraction scheme at each word position $i$ for a given training sentence:

\begin{itemize}
\item[1.] Bias (always active irrespective of input).
\item[2.] Word forms $x_{i-2}, \dots, x_{i+2}$.
\item[3.] Prefixes and suffixes of the word form $x_i$ up to length $\delta_{affix}$.
\item[4.] The brief and long generalized forms of the word forms $x_{i-2}, \dots, x_{i+2}$ following \citet{collins2002}.
\item[5.] The lemmas for word forms $x_{i-2}, \dots, x_{i+2}$ for both TDT and FTB. 
\item[6.] The morphological tags for word forms $x_{i-2}, \dots, x_{i+2}$ for both TDT and FTB. 
\item[6.] The morphological sub-tags for word forms $x_{i-2}, \dots, x_{i+2}$ for both TDT and FTB. 
\end{itemize}

Subsequent to performing the perceptron learning, the resulting classifier can be applied to assign named entity annotation to any given test sentence.


\section{Experiments}
\label{sec: experiments}

\subsection{Data}
\label{sec: data}

\subsection{Reference Systems}
\label{sec: reference systems}

\subsection{Evaluation}
\label{sec: evaluation}

\subsection{Hardware}
\label{sec: hardware}

\subsection{Results}
\label{sec: results}

\subsection{Discussion}
\label{sec: discussion}



\section{Conclusions and Future Work}
\label{sec: conclusions}


Future work: Fine-grained annotation.



%%%%%%%%%%% The bibliography starts:
\newpage
\bibliographystyle{plainnat}
%\bibliographystyle{spbasic}
\bibliography{finer.bib}









\end{document}














